%%% Copyright (C) 2021  Luigi D. C. Soares
%%% 
%%% This program is free software: you can redistribute it and/or modify
%%% it under the terms of the GNU General Public License as published by
%%% the Free Software Foundation, either version 3 of the License, or
%%% (at your option) any later version.
%%% 
%%% This program is distributed in the hope that it will be useful,
%%% but WITHOUT ANY WARRANTY; without even the implied warranty of
%%% MERCHANTABILITY or FITNESS FOR A PARTICULAR PURPOSE.  See the
%%% GNU General Public License for more details.
%%% 
%%% You should have received a copy of the GNU General Public License
%%% along with this program.  If not, see <https://www.gnu.org/licenses/>.

%%%%%%%%%%%%%%%%%%%%%%%%%%%%%%%%%%%%%%%%%%%%%%%%%%%%%%%%%%%%%%%%
%%%%%%%%%%%%%%%%%%%%%%%%%%% Packages %%%%%%%%%%%%%%%%%%%%%%%%%%%
%%%%%%%%%%%%%%%%%%%%%%%%%%%%%%%%%%%%%%%%%%%%%%%%%%%%%%%%%%%%%%%%
\usepackage[utf8]{inputenc}
\usepackage[T1]{fontenc}
\usepackage{bm}
\usepackage{mathtools}
\usepackage{stmaryrd}
\usepackage{minted}
\usepackage{tikz}
\usepackage[style=authoryear, citestyle=authortitle]{biblatex}

%%%%%%%%%%%%%%%%%%%%%%%%%%%%%%%%%%%%%%%%%%%%%%%%%%%%%%%%%%%%%%%%
%%%%%%%%%%%%%%%%%%%%%%% Logo Positioning %%%%%%%%%%%%%%%%%%%%%%%
%%%%%%%%%%%%%%%%%%%%%%%%%%%%%%%%%%%%%%%%%%%%%%%%%%%%%%%%%%%%%%%%
\logo{\begin{tikzpicture}[overlay, remember picture]
    \node[left=0.2cm] at ([yshift=-1cm] current page.north east)
        {\includegraphics
        [width=.125\paperwidth, keepaspectratio]{logo.png}};
\end{tikzpicture}}

%%%%%%%%%%%%%%%%%%%%%%%%%%%%%%%%%%%%%%%%%%%%%%%%%%%%%%%%%%%%%%%%
%%%%%%%%%%%%%%%%%%%%%%%%% Other Settings %%%%%%%%%%%%%%%%%%%%%%%
%%%%%%%%%%%%%%%%%%%%%%%%%%%%%%%%%%%%%%%%%%%%%%%%%%%%%%%%%%%%%%%%
%%% By default, every numbered environment share the same
%%% counter as theorem. We can redefine them so that they
%%% have their own counter. Notice, however, that the Beamer
%%% user guide strongly discourages that, because this way
%%% there might be, for example, a Definition 2 after a
%%% Theorem 10, which (according to them) make things harder
%%% to find. Personally, I disagree; I find it weird to have,
%%% for instance, an Example 2 without an Example 1.
\undef{\example}
\theoremstyle{example}
\newtheorem{example}{Example}
%%% We can also disable numbering entirely:
% \setbeamertemplate{theorems}[default]

%%% Make math font look line in articles.
\usefonttheme[onlymath]{serif}

%%% Math operators.
\DeclareMathOperator{\Def}{\coloneqq}

\makeatletter
\newcommand{\To}{\@ifstar{\@ToB}{\@ToA}}
\newcommand{\@ToA}[1]{\xrightarrow{#1}}
\newcommand{\@ToB}[2]{%
  \xrightarrow{#1}\mathrel{\vphantom{\to}^{#2}}}
\makeatother

%%% Minted config.
\usemintedstyle{monokai}
\renewcommand\theFancyVerbLine{\arabic{FancyVerbLine}}
\setminted[idris]{
    fontfamily=courier,
    frame=lines,
    framesep=2mm,
    breaklines,
    linenos,
    numbersep=5pt,
    stripnl=false,
    autogobble,
    escapeinside=@@
}

%%% Tikz config.
%%% Libraries:
\usetikzlibrary{
    arrows.meta,
    decorations.pathmorphing,
    positioning,
    tikzmark}
%%% TikZ style to apply keys only on specific beamer overlays
%%% only=<overlay spec>{key=value, key=value, ...}
\tikzset{only/.code args={<#1>#2}{%
  \only<#1>{\pgfkeysalso{#2}}%
}}
%%% Default tikz origin coordinates to the center of the page.
%%% Use \shifttikz{<node>}{<shift x>}{<shift y>}{<tikz picture>}
%%% to shift the origin of a tikz picture. For this to work, your
%%% tikz picture must have the following option:
%%% shift = {([xshift = \tikzxshift, yshift = \tikzyshift] \tikzorigin)}
\newcommand{\tikzorigin}{current page.center}
\newlength{\tikzxshift}
\newlength{\tikzyshift}
\setlength{\tikzxshift}{0cm}
\setlength{\tikzyshift}{0cm}
\newcommand{\shifttikz}[4]{{%
    \renewcommand{\tikzorigin}{#1}%
    \setlength{\tikzxshift}{#2}%
    \setlength{\tikzyshift}{#3}%
    #4%
}}

%%% References file.
\addbibresource{references.bib}